\documentclass[13pt]{beamer}

\usetheme{Copenhagen}

\usecolortheme[rgb={0.60884,0.1590909,0.23106061}]{structure} % Red colors
\usefonttheme{serif}
\setbeamerfont{frametitle}{size=\normalsize}
 %\useoutertheme{mathRiponlogo}

%\setbeamertemplate{itemize/enumerate body begin}{\large}
%\setbeamertemplate{itemize/enumerate subbody begin}{\large}
\setbeamertemplate{navigation symbols}{}%remove navigation symbols

\usepackage[english]{babel}
\usepackage[utf8x]{inputenc}
\usepackage{multicol}
\usepackage{fmtcount}

\newcounter{count}
\addtocounter{count}{1}

\newcommand{\question}{ \textbf{(\decimal{count})} \stepcounter{count}}
\newcommand{\pic}[2]{\hfill\includegraphics[scale=#2]{#1}\hspace*{\fill}}

\title[Learning Tips]{How To Study/Learn}
\author{Chester Ismay, Tom Linton}
\institute{Ripon College, Central College}
\date{}

\begin{document}

\begin{frame}
  \titlepage
\end{frame}

%--------------------------------------
\begin{frame}{Learning Quote of the Day}
\Large{``The function of education is to teach one to think intensively and to think critically $\ldots$ intelligence plus character—--that is the goal of true education."\\
\vspace*{0.2in}
\hspace*{0.5in} - Dr. Martin Luther King, Jr.}
\end{frame}

% %--------------------------------------
% \begin{frame}{Beliefs That Make You Stupid}
% 	\question Which of the following was said by Dr. Chew as an \textbf{incorrect} belief about learning?
%     \begin{enumerate}[A]
%     \item Learning is fast and can be done on the first pass.
%     \item Knowledge is not just a bunch of definitions that don't necessarily relate.
%     \item Working hard and committing time and energy is the way to learn.
%     \item Multi-tasking constantly deters your learning.
%     \end{enumerate}
% \end{frame}

%--------------------------------------
\begin{frame}{Metacognition}
	\question Which of the following is FALSE about ``metacognition''?
    \begin{enumerate}[A]
    \item It enables students to better understand how well they know a topic.
    \item Students that perform poorly on assessments tend to be over-confident and have poor metacognition.
    \item Developing better metacognition goes along with improving study skills.
       \item It is the study of how others think.
    \end{enumerate}
\end{frame}

%--------------------------------------
\begin{frame}{Level of Processing/Intention}

\question In general, which of the following combinations learn best (Level of processing/Intention)?
\begin{enumerate}[A]
	\item Deep/Intentional
    \item Deep/Incidental
    \item Shallow/Incidental
    \item Shallow/Intentional
    \item None of the above
\end{enumerate}
\end{frame}

%--------------------------------------
\begin{frame}{Successful Learning}
\question What is the most important factor in successful learning?

\begin{enumerate}[A]
	\item The intention and desire to learn
    \item Paying close attention to the material as you study
    \item Learning in a way that matches your own learning style
    \item The time you spend studying
    \item What you think about while studying
\end{enumerate}
\end{frame}

%--------------------------------------
\begin{frame}{Things That Do Help Learning}
\question Which of the following activities help the learning process?
\begin{enumerate}[A]
	\item Minimizing distractions to maximize focus
    \item Developing accurate metacognition
    \item Deep processing of critical concepts
    \item Practicing retrieval and application
    \item All of the above
\end{enumerate}
\end{frame}

%--------------------------------------
\begin{frame}{Principles for Achieving Deep Processing}
\question Which of the following was not given by Dr. Chew as a deep processing task?
\begin{enumerate}[A]
	\item Elaboration
    \item Distinctiveness
    \item Personal
    \item Automaticity
    \item Retrieval
\end{enumerate}
\end{frame}

% %--------------------------------------
% \begin{frame}{Principles for Achieving Deep Processing}
% \question Overlearning is a good strategy.
% \begin{enumerate}[A]
% 	\item True, overlearning means you are innately good at a subject.
%     \item True, overlearning means you can recall information quickly and easily.
%     \item False, overlearning means that you may be losing some knowledge.
%     \item False, overlearning means that you are wasting your time.
% \end{enumerate}
% \end{frame}

%--------------------------------------
\begin{frame}{Taking Notes In Class}
\question What should you focus on during your professors' lectures?
\begin{enumerate}[A]
	\item Making sure to copy every single thing the professor says into your notes.
    \item Copying only those things that the professor writes on the board since those are the most important ideas.
    \item Thinking about creating memory cues about key relationships in the material.
    \item Lectures aren't important since you can just copy the notes of another classmate. 
\end{enumerate}
\end{frame}

%--------------------------------------
\begin{frame}{Why Not Just Lecture?}
\question Why doesn't Chester just lecture in this class?
\begin{enumerate}[A]
	\item He is boring.
    \item He's lazy; if he doesn't lecture, he doesn't have to do much work as a professor.
    \item He views statistics as a bunch of formulas that you should memorize outside of class if you want to succeed.
    \item He wants to use the most effective strategies possible that are scientifically proven to engage students actively in their learning.
\end{enumerate}
\end{frame}

%--------------------------------------
\begin{frame}{Peer Instruction}
\question Which of the following is NOT true of peer instruction?
\begin{enumerate}[A]
	\item It has been shown to triple the learning gains of students over standard lecture classes.
    \item After you have just learned something is the best time to try to teach it to someone else.
    \item Teaching and talking about topics with your peers is an incredibly valuable and often time-saving way to learn new material.
        \item There is no value.  I just want to listen to lecture passively every class period because that is best for my learning.
\end{enumerate}
\end{frame}

%--------------------------------------
\begin{frame}{Key Terms to Understand in Chapter P}
\begin{multicols}{2}
\begin{itemize}
	\item Data
	\item Variable
    \item Categorical variable
    \item Quantitative variable
    \item Observational unit \\
    \item Distribution
    \item Center
    \item Variability \\
    \item Simulation
    \item Probability
\end{itemize}
\end{multicols}
\end{frame}

\end{document}