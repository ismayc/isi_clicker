\documentclass[13pt]{beamer}

\usetheme{Copenhagen}

\usecolortheme[rgb={0.60884,0.1590909,0.23106061}]{structure} % Red colors
\usefonttheme{serif}
\setbeamerfont{frametitle}{size=\normalsize}
 %\useoutertheme{mathRiponlogo}

%\setbeamertemplate{itemize/enumerate body begin}{\large}
%\setbeamertemplate{itemize/enumerate subbody begin}{\large}
\setbeamertemplate{navigation symbols}{}%remove navigation symbols

\usepackage[english]{babel}
\usepackage[utf8x]{inputenc}
\usepackage{multicol}
\usepackage{fmtcount}

\newcounter{count}
\addtocounter{count}{1}

\newcommand{\quotes}[2]{\centering \Large{``#1"\\
\vspace*{0.2in}
\hspace*{0.5in} - #2}}

\newcommand{\question}{ \textbf{(\decimal{count})} \stepcounter{count}}
\newcommand{\pic}[2]{\hfill\includegraphics[scale=#2]{#1}\hspace*{\fill}}

\title[Chapter P]{Chapter P}
\author{Chester Ismay, Tom Linton}
\institute{Ripon College, Central College}
\date{}

\begin{document}

\begin{frame}
  \titlepage
\end{frame}

%\section{Chapter P}

\begin{frame}{Learning Quote of the Day}
\quotes{Tell me and I forget.\\ Teach me and I remember.\\ Involve me and I learn.}{Benjamin Franklin}
\end{frame}

\begin{frame}{Six Steps of a Statistical Investigation}

\question Put the following steps in the correct chronological order.

\begin{multicols}{2}
\begin{enumerate}
	\item[1] Explore the data.
    \item[2] Formulate conclusions.
    \item[3] Look back and ahead.
    \item[4] Ask a research question.
    \item[5] Design a study and collect data.
    \item[6] Draw inferences beyond the data.
\end{enumerate}
\end{multicols}

\begin{enumerate}[A]
	\item $1 \rightarrow 5 \rightarrow 4 \rightarrow 3 \rightarrow  2 \rightarrow 6$
    \item $1 \rightarrow 4 \rightarrow 5 \rightarrow 6 \rightarrow 2 \rightarrow 3$ 
    \item $4 \rightarrow 5 \rightarrow 1 \rightarrow 6 \rightarrow 2 \rightarrow 3$
    \item $4  \rightarrow 5 \rightarrow 2 \rightarrow 1  \rightarrow 6 \rightarrow 3$
    \item None of the above.
\end{enumerate}

\end{frame}

%----------------------------------------
\begin{frame}

\question Match the following terms with their correct definitions.

\begin{enumerate}
	\item[1] A recorded characteristic on the elements of data
    \item[2] An individual entity on which data is recorded
    \item[3] A pattern of value/category outcomes 
\end{enumerate}

\begin{enumerate}[A]
	\item 1 = observational unit, 2 = distribution, 3 = variable
    \item 1 = distribution, 2 = variable, 3 = observational unit
    \item 1 = variable, 2 = distribution, 3 = observational unit
    \item 1 = variable, 2 = observational unit, 3 = data
    \item None of the above.
\end{enumerate}

\end{frame}

%----------------------------------------
\begin{frame}{Observational Unit \& Variable}

\question \textbf{Select the correct pairing of observational unit (OU) with variable in the following statement.}

\vspace{0.1in}

How much did a typical American consumer spend on Christmas presents in 2012?

\begin{enumerate}[A]
	\item OU = Christmas presents, Variable = Number of people
    \item OU = dollars spent, Variable = Number of family members
    \item OU = dollars spent, Variable = Social class
    \item OU = consumers, Variable = dollars spent on Christmas presents in 2012
    \item None of the above.
\end{enumerate}

\end{frame}


%----------------------------------------
\begin{frame}{Observational Unit \& Variable}
\small{An article in a 2006 issue of the Journal of Behavioral Decision Making reports on a study involving 47 undergraduate students at Harvard. All of the participants were given \$50, but some (chosen at random) were told that this was a “tuition rebate,” while the others were told that this was ``bonus income.” After one week, the students were contacted again and asked how much of the \$50 they had spent and how much they had saved. Researchers wanted to know whether those receiving the “rebate” would tend to save more money than those receiving the ``bonus".}\\

\question \textbf{Select the correct pairing:}

\begin{enumerate}[A]
	\item OU = students, Number of variables = 1
    \item OU = dollars spent, Variable = students
    \item OU = dollars spent, Variable = Rebate/Bonus
    \item OU = students, Only variable = dollars spent
    \item None of the above.
\end{enumerate}

\end{frame}

%----------------------------------------
\begin{frame}{Observational Unit \& Variable}

Do college students who pull all-nighters tend to have lower grade point averages than those who do not pull all-nighters?

\vspace*{0.1in}

\question \textbf{Select the correct pairing:}

\begin{enumerate}[A]
	\item OU = GPA, Only variable = College students
    \item OU = college students, Variables = one quantitative, one categorical
    \item OU = amount of sleep, Only variable = GPA
    \item OU = students who pull all-nighters vs students who don't, Only variable = receiving low GPA
    \item None of the above.
\end{enumerate}

\end{frame}

%----------------------------------------
\begin{frame}%

\frametitle{Problem Statement}%


Statistical evidence played an important role in the murder trial involving
Kristen Gilbert, a nurse who was accused of murdering hospital patients by
giving them fatal doses of heart stimulant. The following table summarizes
eighteen months of data collected from the hospital where she worked. The data
recorded included whether Gilbert was working during an eight-hour shift and
whether a death occurred on the shift:%
\[%
\begin{tabular}
[c]{c|c|c||c|}\cline{2-4}
& Gilbert On Shift & Gilbert Not On & \textbf{Total}\\\hline
\multicolumn{1}{|c|}{Death Occurred} & 40 & 34 & 74\\\hline
\multicolumn{1}{|c|}{No Death Occurred} & 217 & 1350 & 1567\\\hline\hline
\multicolumn{1}{|c|}{\textbf{Total}} & 257 & 1384 & 1641\\\hline
\end{tabular}
\
\]%
\end{frame}%


%----------------------------------------
\begin{frame}%
\frametitle{Observational Units}%

\question What are the observational units?

\begin{enumerate}[A]
	\item Deaths
	\item Hospitals
	\item Nurses
	\item Patients
	\item Shifts
\end{enumerate}


\end{frame}%

%----------------------------------------
\begin{frame}
\frametitle{Variables}%

\question One of the variables of interest in this study is whether or not
Kristen Gilbert was working, what is the other variable and is it quantitative
or categorical?

\begin{enumerate}[A]
	\item If a death occurred on a shift or not, categorical
	\item How many deaths occurred, quantitative
	\item The proportion of shifts with a death, quantitative
	\item Whether a patient lived or died before leaving the hospital, categorical
	\item Whether or not more patients died while Gilbert was working, categorical
\end{enumerate}

\end{frame}%

%----------------------------------------
\begin{frame}%
\frametitle{Center and Variability}%

Consider three distributions of scores for a class of six students.%

\begin{center}
\includegraphics[
natheight=1.175600in,
natwidth=3.472000in,
height=1.247in,
width=3.6264in
]%
{quizscores.PNG}%
\end{center}

\question Without doing any calculations, decide how the means of the three
quizzes compare.

\begin{enumerate}[A]
	\item $Mean(A)=Mean(B)=Mean(C)$
	\item $Mean(B)<Mean(C)<Mean(A)$
	\item $Mean(B)<Mean(A)=Mean(C)$
	\item $Mean(B)<Mean(A)<Mean(C)$
	\item None of the above.
\end{enumerate}

\end{frame}%

%----------------------------------------
\begin{frame}%

\frametitle{Center and Variability}%

Consider three distributions of scores for a class of six students.%


\begin{center}
\includegraphics[
natheight=1.175600in,
natwidth=3.472000in,
height=1.247in,
width=3.6264in
]%
{quizscores.PNG}%
\end{center}


\question Without doing any calculations, decide how the standard deviations
(SD) of the three quizzes compare.

\begin{enumerate}[A]
	\item $SD(A)<SD(B)<SD(C)$
	\item $SD(A)<SD(C)<SD(B)$
	\item $SD(B)<SD(C)<SD(A)$
	\item They are all about the same
	\item None of the above.
\end{enumerate}

\end{frame}


%--------------------------------------
\begin{frame}{Key Terms to Understand in Section 1.1}
\begin{multicols}{2}
\begin{itemize}
	\item Statistical significance
	\item Sample and statistic
    \item Parameter
    \item Chance model
    \item Six Steps of the Statistical Investigation 
    \item 3S Strategy for Measuring Strength of Evidence
\end{itemize}
\end{multicols}
\end{frame}

\end{document}