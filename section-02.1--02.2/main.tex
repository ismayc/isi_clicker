\documentclass[13pt]{beamer}

\usetheme{Copenhagen}

\usecolortheme[rgb={0.60884,0.1590909,0.23106061}]{structure} % Red colors
\usefonttheme{serif}
\setbeamerfont{frametitle}{size=\normalsize}

\setbeamertemplate{navigation symbols}{}%remove navigation symbols

\usepackage[english]{babel}
\usepackage[utf8x]{inputenc}
\usepackage{multicol}
\usepackage{fmtcount}

\newcounter{count}
\addtocounter{count}{1}

\newcommand{\quotes}[2]{\centering \Large{``#1"\\
\vspace*{0.2in}
\hspace*{0.5in} - #2}}

\newcommand{\question}{ \textbf{(\decimal{count})} \stepcounter{count}}
\newcommand{\pic}[2]{\hfill\includegraphics[scale=#2]{#1}\hspace*{\fill}}

\newenvironment{stepenumerate}{\begin{enumerate}[<+->]}{\end{enumerate}}
\newenvironment{stepitemize}{\begin{itemize}[<+->]}{\end{itemize} }

\title{Sections 2.1 and 2.2}
\author{Chester Ismay, Tom Linton}
\institute{Ripon College, Central College}
\date{}


\begin{document}

\begin{frame}
  \titlepage
\end{frame}


\begin{frame}{Learning Quote of the Day}
\quotes{You don't learn to walk by following rules. You learn by doing, and by falling over.}{Richard Branson}
\end{frame}

\begin{frame}{Description}

%Print this for students?

Historically, the average claim size for residential home damage from a hurricane is \$25,300. Hurricane Andrew swept through southern Florida causing billions of dollars in home damage. Because of the severity of the storm and the type of residential construction used in this semi-tropical area, there was concern that the average claim size would be greater than the historical average. Several insurance companies collaborated in a data gathering experiment. They randomly selected 45 homes and found that the average claim for the 45 homes was \$26,500 with a standard deviation of \$6635. Is there good evidence that the average claim for home damage from Hurricane Andrew was greater than the historical average?
\end{frame}

\begin{frame}{Observational Units}
\question What are the observational units?
\begin{enumerate}[A]
   \item Dollar amounts of the claims
   \item Homes damaged by all hurricanes
   \item Hurricanes in residential areas
   \item Insurance claims for home damage due to Andrew%Correct
   \item Insurance companies that cover home damage
\end{enumerate}
\end{frame}

\begin{frame}{Variable of Interest}
\question What is the variable of interest?
\begin{enumerate}[A]
   \item The dollar amount of the claims.%Correct
   \item The average claim size for Hurricane Andrew.
   \item The proportion of homes damaged by Hurricane Andrew.
   \item If claims for home damage were greater for Hurricane Andrew.
   \item Whether or not a home was damaged by Hurricane Andrew.
\end{enumerate}
\end{frame}

\begin{frame}{Parameter or Statistic}
Historically, the average claim size for residential home damage from a hurricane is \$25,300. Hurricane Andrew swept $\ldots$\smallskip

\question Is the value \$25,300 above a hypothesized parameter value or a statistic?
\begin{enumerate}[A]
   \item Hypothesized parameter value%Correct
   \item Statistic
   \item Both
   \item Neither
   \item I am not sure
\end{enumerate}
\end{frame}

\begin{frame}{Parameter or Statistic}
$\ldots$ Several insurance companies collaborated in a data gathering experiment. They randomly selected 45 homes and found that the average claim for the 45 homes was \$26,500 with a standard deviation of \$6635. \smallskip

\question Is the value \$26,500 above a parameter or a statistic? What symbol do we use for numbers like this?
\begin{enumerate}[A]
   \item Parameter, $\mu$
   \item Parameter, $\overline{x}$
   \item Statistic, $\hat{p}$
   \item Statistic, $t$
   \item Statistic, $\overline{x}$%Correct
\end{enumerate}
\end{frame}

\begin{frame}{Hypotheses}
Historically, the average claim size for residential home damage from a hurricane is \$25,300. Hurricane Andrew swept through $\ldots$  the average claim for the 45 homes was \$26,500 with a standard deviation of \$6635. Is there good evidence that the average claim for home damage from Hurricane Andrew was greater than the historical average?\smallskip

\question What is the null hypothesis?
\begin{enumerate}[A]
   \item The long run probability that a home was damaged by Hurricane Andrew is no different than the historical probability.
   \item $H_0:\pi= $ \$25,300
   \item $H_0:\mu= $ \$26,500
   \item The mean claim for home damage from Hurricane Andrew is no different than the historical average.%Correct
   \item $H_a:\overline{x}> $ \$25,300
\end{enumerate}
\end{frame}

\begin{frame}{Hypotheses}
Historically, the average claim size for residential home damage from a hurricane is \$25,300. Hurricane Andrew swept through $\ldots$  the average claim for the 45 homes was \$26,500 with a standard deviation of \$6635. Is there good evidence that the average claim for home damage from Hurricane Andrew was greater than the historical average?\smallskip

\question What is wrong with the following alternative hypothesis?\quad 
$H_a:\pi \neq $ \$26,500
\begin{enumerate}[A]
   \item The parameter is a mean not a proportion.
   \item The research question suggests a right--tailed, not two--tailed alternative.
   \item Hypothesized parameter values, not the values of statistics, should appear in the hypotheses.
   \item All of the above.%Correct
   \item B and C, but not A.
\end{enumerate}
\end{frame}

% \begin{frame}{Test Statistic}
% \question The test statistic for the Hurricane Andrew study is
% $$t=\frac{\overline{x}-\mu_0}{s / \sqrt[]{n}}=
% \frac{26,500-25,300}{6635 / \sqrt[]{45}}\approx 1.21$$
% Assuming ``everything'' is the same, how does the evidence from this $t$ test statistic compare to the evidence from the test statistic $z=1.21$?
% \begin{enumerate}[A]
%    \item The evidence is exactly the same.
%    \item The evidence from $t=1.21$ is slightly stronger than the evidence from $z=1.21$.
%    \item The evidence from $t=1.21$ is slightly weaker than the evidence from $z=1.21$.%Correct
%    \item There is no way to compare the evidence from the two test statistics.
% \end{enumerate}
% \end{frame}

\begin{frame}{$p$--value}
\question Given the test statistic $t=1.21$ for Hurricane Andrew claims, what sort of $p$--value would you expect for this study?

The $p$--value would be
\begin{enumerate}[A]
   \item very small, claims for Andrew were $1.21$ times larger than the historical average.
   \item very small, Andrew's claims were \$1200 above the historical average.
   \item small to moderate, $\overline{x}$ is only $1.21$ SEs above the historical average.%Correct
   \item moderate to large, $t=1.21$ gives little to no evidence against $H_0$.
  % \item large, we have good evidence against $H_0$.
   \item large, 26,500-25,300 = 1200 is very small compared to $s=6635$.
\end{enumerate}
\end{frame}

\begin{frame}{Description}
%
% Print this?
%
A zoologist at a large metropolitan zoo is concerned about a potential new disease present among the 243
sharks living in the large aquarium at the zoo. The zoologist takes a random sample of 15 sharks from the
aquarium, temporarily removes the sharks from the tank and tests them for the disease. He finds that 3 of
the sharks have the disease.
\end{frame}

\begin{frame}{Representative Sample}
\question Is it reasonable to assume that the sample of 15 sharks is a good representation of all 243 sharks in the aquarium?
\begin{enumerate}[A]
   \item No, 15 is too small of a sample for this setting.
   \item No, the population is not 20 times the sample size.
   \item No, a larger convenience sample would be more representative.
   \item Yes, the sample was random.% Correct
   \item Yes, the population is more than 10 times the sample size.
\end{enumerate}
\end{frame}

% \begin{frame}{Validity Conditions}
% \question If we wanted to test if less than 1/4 of the sharks in the aquarium had the disease, are the validity conditions for a theory-based test met?
% \begin{enumerate}[A]
%    \item No, we only have 3 successes.%Correct probably
%    \item No, our sample could be biased.
%    \item Yes, the sample was random.
%    \item Yes, we have a process so everything is fine.
%    \item Yes, the population size is between 10 and 20 times the sample size.
% \end{enumerate}
% \end{frame}

\begin{frame}{Hypotheses}
\question Recall that the zoologist found $3/15=20\%$ of the sharks in the sample with the disease. If he wanted to test if less than 1/4 of the sharks in the aquarium had the disease, what is his null hypothesis?
\begin{enumerate}[A]
   \item $H_0=0.25$
   \item $H_0:\mu=0.2$
   \item $H_0:\pi=0.25$ %Correct
   \item $H_0:\hat{p}<0.25$
   \item $H_0:\overline{x}=3/15$
\end{enumerate}
\end{frame}

\begin{frame}{$p$--value}
\question A simulation based $p$--value for this test is $p=0.48$. Which of the following are correct conclusions based on this $p$--value?
\begin{enumerate}[A]
   \item We have good evidence that $\pi=0.25$.
   \item We do not have good evidence that $\pi<0.25$. 
   \item $0.25$ is one of many plausible values for $\pi$.
   %\item All of the above.
   \item A and B but not C.
   \item B and C but not A.  %Correct
\end{enumerate}
\end{frame}
   

\begin{frame}{$p$--value}
\question A simulation based $p$--value for this test is $p=0.48$. Which of the following are correct statements about this $p$--value?
\begin{enumerate}[A]
   \item When $\pi=.25$, random samples of size 15 will have $\hat{p}\leq \frac{3}{15}$ about 48\% of the time.% Correct
   \item The sample must have been biased because $\hat{p}<0.25$, so we should have a small $p$--value.
   \item The simulation must have been done wrong, the $p$--value should be smaller.
   \item If we took a larger sample ($n>15$), but still had $\hat{p}=0.2$, the $p$--value would be larger.
   \item If the population size were doubled (486 sharks in the aquarium), but everything else stayed the same, the $p$--value would decrease.
  % \item None of the above.
\end{enumerate}
\end{frame}

\begin{frame}{Shape}
\question Here is a dot plot for the ages of 21 male rattlesnakes captured at a single site. Assume that these 21
snakes can be regarded as a random sample of all male rattlesnakes at that site. The average age is 8.571 years,
with a standard deviation of 2.942 years. Describe the shape of this distribution.
\begin{multicols}{2}
\begin{enumerate}[A]
   \item skewed left
   \item slightly skewed left%Correct
   \item fairly symmetric%OK
   \item slightly skewed right
   \item skewed right
\end{enumerate}
\pic{rattlesnakeweights.PNG}{.55}
\end{multicols}
\end{frame}

\begin{frame}{Measures of Center}
\question Based on the dot plot of rattlesnake ages below, how would you expect the median to compare to the mean?\smallskip

The median would be
\begin{multicols}{2}
\begin{enumerate}[A]
   \item less than the mean%Correct
   \item roughly equal to the mean%OK
   \item greater than the mean
   \item You cannot tell
\end{enumerate}
\pic{rattlesnakeweights.PNG}{.55}
\end{multicols}
\end{frame}

\begin{frame}{Median}
\question Based on the dot plot of rattlesnake ages below, estimate the value of the median
\begin{multicols}{2}
\begin{enumerate}[A]
   \item 7
   \item 8
   \item 9%Correct
   \item 11%(21+1) / 2 =11
   \item You cannot tell
\end{enumerate}
\pic{rattlesnakeweights.PNG}{.55}
\end{multicols}
\end{frame}

\begin{frame}{Standard Deviation}
\question If the ages of the three youngest rattlesnakes were all changed to 0, what would happen to the SD?
%\begin{multicols}{2}
\pic{rattlesnakeweights.PNG}{.65}
\begin{enumerate}[A]
   \item It would get smaller
   \item It would not change
   \item It would get larger%Correct
   \item You cannot tell
\end{enumerate}

%\end{multicols}
\end{frame}

\begin{frame}{Mean}
\question If the ages of the three youngest rattlesnakes were all changed to 0, what would happen to the mean?
%\begin{multicols}{2}
\pic{rattlesnakeweights.PNG}{.65}
\begin{enumerate}[A]
   \item It would get smaller%Correct
   \item It would not change
   \item It would get larger
   \item You cannot tell
\end{enumerate}

%\end{multicols}
\end{frame}

\begin{frame}{Median}
\question If the ages of the three youngest rattlesnakes were all changed to 0, what would happen to the median?
%\begin{multicols}{2}
\pic{rattlesnakeweights.PNG}{.65}
\begin{enumerate}[A]
   \item It would get smaller
   \item It would not change%Correct
   \item It would get larger
   \item You cannot tell
\end{enumerate}

%\end{multicols}
\end{frame}

\begin{frame}{Rarity}
\question For the 21 rattlesnakes, the mean age was $8.571$ years with $s=2.942$ years. Which of the following calculations do you feel gives the best measure of how rare a $12.5$ year old rattlesnake would be at this location?
%\begin{multicols}{2}
\pic{rattlesnakeweights.PNG}{.5}
\begin{enumerate}[A]
   \item It is the 3rd oldest rattlesnake.
   \item It is $12.5-8.571\approx 3.9$ years above average.
   \item It is approximately $\frac{12.5}{2.942}\approx 4.2$ SEs above zero.
   \item It's approximate standardized value is $\frac{12.5-8.571}{2.942}\approx 1.34$.%Correct
\end{enumerate}

%\end{multicols}
\end{frame}


\begin{frame}{Key Terms and Ideas to Understand in Section 2.3}
%\begin{multicols}{2}
\begin{itemize}
	\item Significance level
    \item Type I error (false alarm)
    \item Type II error (missed opportunity)
    \item Power of a test
\end{itemize}
%\end{multicols}
\end{frame}


\end{document}