\documentclass[13pt]{beamer}

\usetheme{Copenhagen}

\usecolortheme[rgb={0.60884,0.1590909,0.23106061}]{structure} % Red colors
\usefonttheme{serif}
\setbeamerfont{frametitle}{size=\normalsize}

\setbeamertemplate{navigation symbols}{}%remove navigation symbols

\usepackage[english]{babel}
\usepackage[utf8x]{inputenc}
\usepackage{multicol}
\usepackage{fmtcount}

\newcounter{count}
\addtocounter{count}{1}

\newcommand{\quotes}[2]{\centering \Large{``#1"\\
\vspace*{0.2in}
\hspace*{0.5in} - #2}}

\newcommand{\question}{ \textbf{(\decimal{count})} \stepcounter{count}}
\newcommand{\pic}[2]{\hfill\includegraphics[scale=#2]{#1}\hspace*{\fill}}

\newenvironment{stepenumerate}{\begin{enumerate}[<+->]}{\end{enumerate}}
\newenvironment{stepitemize}{\begin{itemize}[<+->]}{\end{itemize} }
\newenvironment{click}{\begin{enumerate}[A]}{\end{enumerate}}

\title{Sections 8.1 and 8.2 }
\author{Chester Ismay, Tom Linton}
\institute{Ripon College, Central College}
\date{}


\begin{document}

\begin{frame}
  \titlepage
\end{frame}


\begin{frame}{Learning Quote of the Day}
\quotes{Learning is not attained by chance, it must be sought for with ardor and diligence.}{Abigail Adams}

\end{frame}

\begin{frame}{Xylitol Gum and Ear Infections}
%From Utts, Mind on Statistics, 4th edition
Xylitol is a food sweetener that may also have antibacterial properties. A Finland study was designed to test if chewing gum containing xylitol could reduce the risk of middle ear infections in children in daycare centers. The researchers divided a representative group of 533 daycare children into three groups. Group 1 regularly chewed gum containing xylitol, group 2 took xylitol lozenges, and group 3 regularly chewed gum without xylitol. Subjects were followed for three months and for each child the researchers recorded whether or not the child had an ear infection.
\end{frame}

\begin{frame}{Results}
The results are shown in the table below.
\begin{center}
\begin{tabular}
[c]{|c||c|c|c|c|}\hline
Ear Inf? & Lozenge & Xylitol Gum & Placebo & Total\\\hline
Yes & 39 (22.2\%) & 29 (16.2\%) & 49 (27.6\%) & 117 (22\%)\\\hline
No & 137 (77.8\%) & 150 (83.8\%) & 129 (72.5\%) & 416 (78\%)\\\hline
Total & 176 & 179 & 178 & 533\\\hline
\end{tabular}
\end{center}
\end{frame}

\begin{frame}{Observational Units}
\question What are the observational units?
\begin{click}
   \item Gum and lozenges
   \item Ear infections
   \item Finish children in daycare centers% Correct
   \item Whether or not xylitol can reduce the number of ear infections in Finnish daycare children
   \item None of the above
\end{click}
\end{frame}

\begin{frame}{Response Variable}
\question What is the response variable?
\begin{click}
   \item Whether or not a subject developed an ear infection.% Correct
   \item The proportion of children that developed an ear infection.
   \item The treatment given, xyltiol gum, xylitol lozenge, or placebo gum.
   \item Whether or not xylitol can reduce the number of ear infections in Finnish daycare children.
   \item The mean absolute difference in proportions.
\end{click}
\end{frame}

\begin{frame}{Single Summary}
\question If one wanted to summarize the overall proportion of subjects that developed an ear infection, which of the following would be the best statistic to use?
\begin{click}
   \item $\hat{p}_{placebo}$, since this group really did not receive any treatment.
   \item $\hat{p}_{lozenge}$, since it is in between the other two conditional proportions.
   \item The average proportion, $\dfrac{\hat{p}_{lozenge}+\hat{p}_{gum}+\hat{p}_{placebo}}{3}=0.22$.
   \item The total proportion, $\hat{p} = \frac{117}{533} = 0.2195$.
\end{click}
\end{frame}

\begin{frame}{The MAD Statistic}
\question What is the value of the MAD statistic for the xylitol gum study?
\begin{click}
   \item $0.22$% Just the overall proportion, also (p1+p2+p3)/3
   \item $0.075$% Correct
   \item $0.226$% The differences added but not divided by 3
   \item $0.038$% The average deviation from p-hat = 0.22
   \item $0.66$% The sum of the cond. proportions
   \item None of the above.
\end{click}
\end{frame}

\begin{frame}{Null Hypothesis}
\question What would the null hypothesis be if we wanted to test all three proportions to see if they are equal?
\begin{click}
   \item $H_0: \hat{p}_{gum} = \hat{p}_{lozenge} = \hat{p}_{placebo}$
   \item $H_0:MAD = 0.076$
   \item $H_0:\pi_{earInfection} = 0.5$
   \item $H_0:\pi_{earInfection} - \pi_{noInfection} = 0$
   \item $H_0:\pi_{gum} - \pi_{lozenge} - \pi_{placebo} = 0$ 
   \item None of the above% Correct
\end{click}
\end{frame}

\begin{frame}{Alternative Hypothesis}
\question What would the alternative hypothesis be if we wanted to test all three proportions to see if they are equal?
\begin{click}
   \item $H_a:$ there is an association between the treatments and ear infections.
   \item $H_a: \pi_{gum} \neq \pi_{lozenge} \neq \pi_{placebo}$
   \item $H_a:\pi_{earInfection} - \pi_{noInfection} \neq 0$
   \item $H_a:$ at least two of $\pi_{gum}, \pi_{lozenge}$ and $\pi_{placebo}$ are different.
   \item Two of the above are correct.% Correct
\end{click}
\end{frame}

\begin{frame}{Expected Counts}
\question True or False, the expected number of ear infections in the xylitol gum group is $\hat{p}_{earInfection} \times n_{gum} = \frac{117}{533} \times 179 \approx 39.29$ (all the arithmetic is correct).
\begin{click}
   \item True and I am confident
   \item True but I am not too confident
   \item False and I am confident
   \item False but I am not too confident
\end{click}
\end{frame}

\begin{frame}{CI and Expected Counts}
\question Values of the counts $Observed - Expected$ are shown in the table below. The Multiple Proportions applet produced CIs for $\pi_{lozenge} - \pi_{gum}, \pi_{gum} - \pi_{placebo}$ and $\pi_{lozenge} - \pi_{placebo}$ to be (in random order) $(-0.198, -0.028),(-0.144, 0.036)$ and $(-0.022, 0.141)$, which is the CI for $\pi_{lozenge} - \pi_{gum}$?
\smallskip

\begin{columns}[onlytextwidth]
\begin{column}{.33\textwidth}
   \begin{click}
      \item $(-0.198, -0.028)$%No, all negative values
      \item $(-0.144, 0.036)$%No, mostly negative values
      \item $(-0.022, 0.141)$% Correct, mostly positive values
   \end{click}
\end{column}
\begin{column}{.67\textwidth}
      \pic{gumtable.PNG}{0.95}
\end{column}
\end{columns}
\end{frame}


\begin{frame}{$p$--value}
\question The study had $MAD \approx 0.076$; 500 simulated values of the MAD statistic are shown in the image below, what can you say about the $p$--value for this study?
\smallskip

\begin{columns}[onlytextwidth]
\begin{column}{.35\textwidth}
   \begin{click}
      \item $p < 0.01$
      \item $0.01 < p < 0.05$
      \item $0.05 < p < 0.1$% Correct
      \item $0.1 < p < 0.5$
      \item $p > 0.5$
   \end{click}
\end{column}
\begin{column}{.65\textwidth}
      \pic{gumsimul.PNG}{0.95}
\end{column}
\end{columns}
\end{frame}

\begin{frame}
\question The chi-squared contributions $\left ( \frac{(observed - expected)^2}{expected} \right)$ for the xylitol gum group ear infections, and no ear infections are 2.70 and 0.76 respectively, which is the best interpretation of these values?
\begin{click}
   \item Both are much larger than 0.05 and therefore represent no evidence against our null hypothesis.
   \item Both values are larger than expected and the ear infection contribution of 2.70 is more significant than the no infection contribution of 0.76.% Correct
   \item Both values are quite small and are essentially negligible in terms of the total chi-square statistic.
   \item The smaller the contribution, the more evidence against the null hypothesis, so 0.76 is more significant than 2.70.
\end{click}
\end{frame}


\begin{frame}{Key Terms and Ideas to Understand in Chapter 8 }
%\begin{multicols}{2}
\begin{itemize}
	\item Chi-square distribution
    \item Chi-square statistic
    \item MAD derived statistic
    \item Validity conditions for chi-square test
    \item Pairwise confidence intervals
\end{itemize}
%\end{multicols}
\end{frame}


\end{document}