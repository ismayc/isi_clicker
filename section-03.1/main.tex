\documentclass[13pt]{beamer}

\usetheme{Copenhagen}

\usecolortheme[rgb={0.60884,0.1590909,0.23106061}]{structure} % Red colors
\usefonttheme{serif}
\setbeamerfont{frametitle}{size=\normalsize}

\setbeamertemplate{navigation symbols}{}%remove navigation symbols

\usepackage[english]{babel}
\usepackage[utf8x]{inputenc}
\usepackage{multicol}
\usepackage{fmtcount}

\newcounter{count}
\addtocounter{count}{1}

\newcommand{\quotes}[2]{\centering \Large{``#1"\\
\vspace*{0.2in}
\hspace*{0.5in} - #2}}

\newcommand{\question}{ \textbf{(\decimal{count})} \stepcounter{count}}
\newcommand{\pic}[2]{\hfill\includegraphics[scale=#2]{#1}\hspace*{\fill}}

\newenvironment{stepenumerate}{\begin{enumerate}[<+->]}{\end{enumerate}}
\newenvironment{stepitemize}{\begin{itemize}[<+->]}{\end{itemize} }

\title{Section 3.1}
\author{Chester Ismay, Tom Linton}
\institute{Ripon College, Central College}
\date{}


\begin{document}

\begin{frame}
  \titlepage
\end{frame}


\begin{frame}{Learning Quote of the Day}
\quotes{Learning is not child's play; we cannot learn without pain.}{Aristotle}
\end{frame}

\begin{frame}{Where's the $p$-value?}
Let $\pi$ denote some population proportion of interest and suppose a 99\% confidence interval for $\pi$ is calculated to be \\ (0.5, 0.8). Also, suppose that we want to test $H_0: \pi = 0.79$ vs. $H_a: \pi \ne 0.79$. 

\smallskip

\question What can you say for sure about the corresponding $p$-value?

\begin{enumerate}[A]
	\item The corresponding $p$-value will be greater than 5\%.
    \item The corresponding $p$-value will be equal to 1\%.
    \item The corresponding $p$-value will greater than 1\%. %correct
	\item The corresponding $p$-value will less than 1\%.
\end{enumerate}
\end{frame}

\begin{frame}{What's the confidence interval?}
Suppose we are constructing a confidence interval using repeated tests of significance. Using two-sided tests each time with the following null hypotheses, we obtain these $p$-values.

\pic{table.PNG}{1.1}

\question Give an approximate 90\% CI in the form $(lower, upper)$.

\begin{multicols}{2}
\begin{enumerate}[A]
	\item $(0.47, 0.58)$
    \item $(0.47, 0.57)$ %correct
    \item $(0.48, 0.56)$
    \item $(0.46, 0.57)$
\end{enumerate}
\end{multicols}

\end{frame}

\begin{frame}{Plausible values}
Let $\pi$ denote some population proportion of interest and suppose a 95\% confidence interval for $\pi$ is calculated to be (0.25, 0.55).

\question Give one plausible value for $\pi$.

\begin{enumerate}[A]
	\item 0.20
    \item 0.56
    \item 0.13
    \item 0.95
    \item None of the above. %correct
\end{enumerate}
\end{frame}

\begin{frame}{Different hypothesized values}
Let $\pi$ denote some population proportion of interest and suppose a 95\% confidence interval for $\pi$ is calculated to be (0.25, 0.55).

\question Make a guess for what the two-sided $p$-value would be if you hypothesized that $\pi = 0.58$.

\begin{enumerate}[A]
	\item 0.99
    \item 0.01 %correct
    \item 0.25
    \item 0.50
    \item None of the above.
\end{enumerate}
\end{frame}

\begin{frame}{Guessing $p$-values}
A representative sample to estimate some population proportion $\pi$ produces the sample proportion $\hat{p}=0.54$. A test of $H_0:\pi=0.51$ against the alternative $\pi\neq 0.51$ gives a $p$--value of $p=0.16$.\smallskip

\question What can you say about the $p$--value one would obtain for the null hypothesis $H_0:\pi=0.52$?
\begin{enumerate}[A]
   \item It would be smaller than $p=0.16$ because $\hat{p}=0.54$ is closer to $\pi_0=0.52$ than it is to $\pi_0=0.51$.
   \item It would be larger than $p=0.16$ because $\hat{p}=0.54$ is closer to $\pi_0=0.52$ than it is to $\pi_0=0.51$. %correct
   \item It would be the same because $\hat{p}$ has remained the same.
   \item You would have to run the test or a simulation to decide.
\end{enumerate}
\end{frame}

\begin{frame}{Confidence Intervals and $p$--values}
A representative sample to estimate some population proportion $\pi$ produces the sample proportion $\hat{p}=0.54$. A test of $H_0:\pi=0.51$ against the alternative $\pi\neq 0.51$ gives a $p$--value of $p=0.16$.\smallskip

\question What can you say about the $p$--value one would obtain for the null hypothesis $H_0:\pi=0.57$?
\begin{enumerate}[A]
   \item It would be smaller than $p=0.16$ because $\hat{p}=0.54$ is less than $\pi_0=0.57$.
   \item It would be larger than $p=0.16$ because the $z$--score for $0.57$ is positive.
   \item It would be similar to $p=0.16$ because $0.57$ is the same distance from $0.54$ as $0.51$ is. %Correct
   \item You would have to run the test or a simulation to decide.
\end{enumerate}
\end{frame}

\begin{frame}{Confidence Intervals and $p$--values}
A representative sample to estimate some population proportion $\pi$ produces the sample proportion $\hat{p}=0.54$. A test of $H_0:\pi=0.51$ against the alternative $\pi\neq 0.51$ gives a $p$--value of $p=0.16$.\smallskip

\question Assuming a two--tailed alternative, which of the following null hypotheses do you think will give a $p$--value smaller than $p=0.16$?
\begin{enumerate}[A]
   \item $H_0:\pi=0.64$.
   \item $H_0:\pi=0.53$
   \item $H_0:\pi=0.47$.
   \item B and C but not A.
   \item A and C but not B. %correct
   %\item None of the above.
\end{enumerate}
\end{frame}

\begin{frame}{Comparing Confidence Intervals}
Teenage hearing loss has increased significantly in America over the past several years. A 95\% CI using the interval of plausible values method is given by $(0.16,0.23)$.\smallskip

%
%I used 78 / 400 to create the original and rounded
%

\question Which of the following could be the 99\% CI based on the same sample?
\begin{enumerate}[A]
   \item $(0.13,0.22)$%No, misses some values in 95% CI
   \item $(0.18,0.21)$%No, this interval is narrower
   \item $(0.14,0.25)$%Yes
   \item $(0.16,0.27)$%No, wrong center or 0.129 has p value about .05, not .01
   %\item All of the above.
   \item None of the above.
\end{enumerate}
\end{frame}



\begin{frame}{Key Terms and Ideas to Understand in Section 3.2 and 3.3}
\begin{multicols}{2}
\begin{itemize}
	\item Confidence level
    \item Margin-of-error
    \item Center of CI
    \item Multiplier
    \item Standard error of $\hat{p}$
    \item Standard error of $\overline{x}$
    \item 2 SD Method for CIs
\end{itemize}
\end{multicols}
\end{frame}


\end{document}