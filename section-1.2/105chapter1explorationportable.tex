\documentclass[13pt]{beamer}

\usetheme{Copenhagen}

\usecolortheme[rgb={0.60884,0.1590909,0.23106061}]{structure} % Red colors
\usefonttheme{serif}
\setbeamerfont{frametitle}{size=\normalsize}
 %\useoutertheme{mathRiponlogo}

%\setbeamertemplate{itemize/enumerate body begin}{\large}
%\setbeamertemplate{itemize/enumerate subbody begin}{\large}
\setbeamertemplate{navigation symbols}{}%remove navigation symbols

\usepackage[english]{babel}
\usepackage[utf8x]{inputenc}
\usepackage{multicol}
\usepackage{fmtcount}

\newcounter{count}
\addtocounter{count}{1}

\newcommand{\quotes}[2]{\centering \Large{``#1"\\
\vspace*{0.2in}
\hspace*{0.5in} - #2}}

\newcommand{\question}{ \textbf{(\decimal{count})} \stepcounter{count}}
\newcommand{\pic}[2]{\hfill\includegraphics[scale=#2]{#1}\hspace*{\fill}}

\newenvironment{stepenumerate}{\begin{enumerate}[<+->]}{\end{enumerate}}
\newenvironment{stepitemize}{\begin{itemize}[<+->]}{\end{itemize} }

\title{Section 1.2}
\author{Chester Ismay, Tom Linton}
\institute{Ripon College, Central College}
\date{}


\begin{document}

\begin{frame}
  \titlepage
\end{frame}



\begin{frame}{Learning Quote of the Day}
\quotes{For me, I am driven by two main philosophies: know more today about the world than I knew yesterday and lessen the suffering of others. You'd be surprised how far that gets you.}{Neil deGrasse Tyson}
\end{frame}

\begin{frame}{Center of Null Distribution}
Suppose you are testing the hypotheses $H_0 : \pi = 0.25$ and $H_a : \pi < 0.25$ and the observed statistic, $\hat{p}$, is equal to 0.30 with a sample size of 100.

\vspace*{0.1in}

\question If you are using a proportion as your statistic, where do you expect your null distribution to be centered?
\begin{enumerate}[A]
	\item 0.25
    \item 0.50
    \item 0.30
    \item 30
    \item None of the above
\end{enumerate}

\end{frame}

\begin{frame}{Center of Null Distribution (continued)}
Suppose you are testing the hypotheses $H_0 : \pi = 0.25$ and $H_a : \pi < 0.25$ and the observed statistic, $\hat{p}$, is equal to 0.30 with a sample size of 100.

\vspace*{0.1in}

\question If you are using a count as your statistic, where do you expect your null distribution to be centered?
\begin{enumerate}[A]
	\item 25
    \item 50
    \item 0.30
    \item 30
    \item None of the above
\end{enumerate}

\end{frame}


\begin{frame}{$P$-value}
\question Which of the following is the correct definition of ``$p$-value"?
\begin{enumerate}[A]
	\item The proportion of successes in our sample
    \item The probability the null hypothesis is true
    \item The proportion of times the observed statistic occurs out of the total number of replications in the null distribution
    \item The chance of obtaining a sample statistic as extreme or more extreme than the observed statistic, assuming the alternative hypothesis is true
    \item None of the above
\end{enumerate}
\end{frame}

\begin{frame}{Form of the Null Hypothesis}
\question Which of the following is the correct general form for the null hypothesis when testing one proportion?
\begin{enumerate}[A]
	\item $H_0: \pi = 0.5$
    \item $H_0: \hat{p} = \text{hypothesized parameter value}$
    \item $H_0: \pi =  \text{hypothesized parameter value}$
    \item $H_0: \pi = \hat{p}$
    \item $H_0: \pi \geq \text{hypothesized parameter value}$
\end{enumerate}

\end{frame}

\begin{frame}{Which Tire? Setup}

A legendary story on college campuses concerns two students who miss
a chemistry exam because of excessive partying but blame their absence on a
flat tire. The professor allows them to take a make-up exam, and sends them to
separate rooms to take it. The first question, worth five points, is quite
easy. The second question, worth ninety-five points, asks: ``Which tire was
it?"
\end{frame}

\begin{frame}{Ask a research question}

Do students pick which tire went flat in equal proportions? It has
been conjectured that when students are asked this question and forced to give
an answer (left front, left rear, right front, or right rear) off the top of
their head, they tend to answer \textquotedblleft right
front\textquotedblright\ more than would be expected by random chance.
\end{frame}

\begin{frame}{Design a study and collect data}

To test this conjecture about the right front tire, a recent class
of 28 students was asked if they were in this situation, which tire would they
say had gone flat. We obtained the following results:%

\[%
\begin{tabular}
[c]{|c|c|c|c|}\hline
Left front & Left rear & Right front & Right rear\\\hline
6 & 4 & 14 & 4\\\hline
\end{tabular}
\]
\end{frame}

\begin{frame}{Observational Units}
\question What are the observational units?
\begin{enumerate}[A]
   \item The students.%Correct
   \item The four tires.
   \item Whether students select \textit{right front} more than expected.
   \item Whether or not students select \textit{right front}.
\end{enumerate}
\end{frame}

\begin{frame}{Variable of Interest}
\question What is the variable of interest in this study?
\begin{enumerate}[A]
   \item The long run probability that a student picks the right front tire. 
   \item The proportion of students who picked the right front tire.
   \item Which tire each student picked.%Correct
   \item How many students picked each tire.
\end{enumerate}
\end{frame}

\begin{frame}{Parameter}
\question What is the parameter of interest in this study?
\begin{enumerate}[A]
   %\item How many students picked each tire.
   \item The proportion of students in the class who picked the right front tire.
   \item Which tire each student picked.
   \item The long run probability that a student picks the right front tire.%Correct
   \item The long run probability that more than 1/4 of the students will pick the right front tire.
   \item $\hat{p}=\frac{14}{28}$.
\end{enumerate}
\end{frame}

\begin{frame}{Hypotheses}
\question What is the appropriate null hypothesis for this setting?
\begin{enumerate}[A]
   \item $H_0:\pi=0.5$
   \item $H_a:\pi>0.25$
   \item $H_0:\hat{p}=0.5$
   \item In the long run, students will pick each tire equally often.%Correct, but this is stronger than pi=.25
   \item In the long run, more students than chance predicts will pick the right front tire. 
\end{enumerate}
\end{frame}

\begin{frame}{$p$--value}

\question Recall that the study had $\hat{p}=\frac{14}{28}$ students pick the right front tire. Shown below are 100 simulated values for $X=$ the number of students (out of 28) who picked the right front tire, assuming $\pi=0.25$. Use the results to estimate the $p$--value for this test of significance.

\begin{multicols}{2}

\begin{enumerate}[A]
   \item $0$
   \item $0.01$%correct
   \item $0.03$
   \item $0.58$
   %\item None of the above
\end{enumerate}

\pic{tiresimulation.PNG}{.9}
\end{multicols}

\end{frame}

\begin{frame}{Exploration 1.3 Facial Prototyping}

\begin{stepitemize}
   \item Do we have preconceived notions of what persons with a certain common name look like? 
   \item For example, do people automatically construct an image in their head for a name like Sarah or Brian?
   \item You will soon be shown a picture of two men named Bob and Tim, and you will be asked to decide who is on the left, Bob or Tim, based solely on the photos.
\end{stepitemize}

\end{frame}

\begin{frame}{Data Collection Section 1.3}
\question Who is on the left?
\pic{timandbob.jpg}{.6}
\begin{enumerate}[A]
   \item Bob
   \item Tim%more should chose Tim
\end{enumerate}

\end{frame}

\begin{frame}{Key Terms and Ideas to Understand in Sections 1.3 and 1.4}
%\begin{multicols}{2}
\begin{itemize}
	\item Standardized statistic ($z$--score)
    \item The mean of the null distribution
    \item The standard deviation (SD) of the null distribution
    \item One- or two-sided test
    \item $p$-value
\end{itemize}
%\end{multicols}
\end{frame}


\end{document}