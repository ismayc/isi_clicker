\documentclass[13pt]{beamer}

\usetheme{Copenhagen}

\usecolortheme[rgb={0.60884,0.1590909,0.23106061}]{structure} % Red colors
\usefonttheme{serif}
\setbeamerfont{frametitle}{size=\normalsize}

\setbeamertemplate{navigation symbols}{}%remove navigation symbols

\usepackage[english]{babel}
\usepackage[utf8x]{inputenc}
\usepackage{multicol}
\usepackage{fmtcount}

\newcounter{count}
\addtocounter{count}{1}

\newcommand{\quotes}[2]{\centering \Large{``#1"\\
\vspace*{0.2in}
\hspace*{0.5in} - #2}}

\newcommand{\question}{ \textbf{(\decimal{count})} \stepcounter{count}}
\newcommand{\pic}[2]{\hfill\includegraphics[scale=#2]{#1}\hspace*{\fill}}
\newcommand{\blank}[2]{\underline{\invisible<#2>{{\color{red} {#1}}}}}

\newenvironment{stepenumerate}{\begin{enumerate}[<+->]}{\end{enumerate}}
\newenvironment{stepitemize}{\begin{itemize}[<+->]}{\end{itemize} }
\newenvironment{click}{\begin{enumerate}[A]}{\end{enumerate}}

\title{Section 5.3}
\author{Chester Ismay, Tom Linton}
\institute{Ripon College, Central College}
\date{}


\begin{document}

\begin{frame}
  \titlepage
\end{frame}


\begin{frame}{Learning Quote of the Day}
\quotes{There is no secret to success, but it's the result of hard work, good preparation and learning from failure.}{Ally Mbululo}
\end{frame}

\begin{frame}{Conserving Hotel Towels?}
Many hotels have begun a
conservation program that encourages guests to re-use towels rather than
have them washed on a daily basis. A recent study examined whether one
method of encouragement might work better than another. Different signs
explaining the conservation program were placed in the bathrooms of the
hotel rooms, with random assignment determining which rooms received which
sign. One sign mentioned the importance of environmental protection, whereas
another sign claimed that 75\% of the hotel's guests choose to participate
in the program. The researchers suspected that the latter sign, by appealing
to a social norm, would produce a higher proportion of hotel guests who
agree to re-use their towels. Researchers used the hotel staff (a mid--sized,
mid--priced hotel in the Southwest that was part of a well-known national
hotel chain) to record whether guests staying for multiple nights agreed to
reuse their towel after the first night.
\end{frame}

\begin{frame}{Observational Units}
\question What are the observational units?
\begin{click}
   \item Guests at hotels
   \item Mid--sized mid--priced hotels
   \item Multi-night guests at hotels%Correct
   \item The two signs
   \item Towels
   %\item 
\end{click}
\end{frame}

\begin{frame}{Explanatory Variable}
\question What is the explanatory variable?
\begin{click}
   \item If a guest stayed multiple nights.
   \item Which sign (social norm or environmental) was displayed.%Correct
   \item If the social norm sign produced more towel reuse than the environmental sign.
   \item Whether or not a guest chose to re--use their towels.
   \item None of the above.
\end{click}
\end{frame}

\begin{frame}{Study Results}
The study results are summarized in the table below. Assume we abbreviate the social norm sign group using ``sn'' and the environmental sign group using ``e''.
\[
\begin{tabular}{|c||c|c|c|}
\hline
& Social norm &Environmental & Total \\ \hline\hline
Re-used Towel & 98 & 74 & 172 \\ \hline
Did Not Re-use & 124 & 137 & 261 \\ \hline
Total & 222 & 211 & 433 \\ \hline
\end{tabular}%
\]
We see that
\begin{itemize}
   \item  the statistic is \blank{$\hat{p}_{sn} - \hat{p}_e = 0.441 - 0.351 = 0.09$;}{1} \pause
   \item the overall or combined proportion is $\dfrac{172}{433} \approx 0.397$.
\end{itemize}
\end{frame}

\begin{frame}{Significance}
\question Would you guess that the difference in conditional proportions, $\hat{p}_{sn} - \hat{p}_e \approx 0.09$ is statistically significant?
\begin{click}
%
% No correct answer here, just an opinion
%
   \item No, $p=0.09$ does not give strong evidence against the null hypothesis.
   \item No, a 9\% difference is small compared to 50\%.
   \item No, but not for either reason above.
   \item Yes, the sample sizes are fairly large and the difference is relatively big.
   \item Yes, but not for the reason above.
   %\item I am not sure.
\end{click}
\end{frame}

\begin{frame}{Null Hypothesis}
\question What is the most appropriate null hypothesis?
\begin{click}
   %\item $H_0: \pi =0.5$
   \item $H_0: \pi = 0.397$
   \item $H_0: \hat{p}_{sn} - \hat{p}_e = 0$
   \item The social norm sign has a 50--50 chance of having a guest re--use their towel.
   \item The social norm sign increases the probability of towel re--use.
   \item None of the above.%Correct
\end{click}
\end{frame}

\begin{frame}{Parameter}
\question The correct alternative hypothesis could be stated as $H_a: \pi_{sn} - \pi_e > 0$. Which is the best description of the parameter $\pi_{sn}$?
\begin{click}
   \item The long run probability that a guest with the social norm sign opts to re--use their towels.%Correct
   \item $\pi_{sn} = \frac{98}{222} \approx 0.441$.
   \item The population increase in the proportion of guests opting to re--use their towels with the social norm sign, as opposed to the environmental sign.
   \item The population proportion of guests who opt to re--use their towels.
   \item The probability that the social norm sign will increase the proportion of guests who opt to re--use their towels (over the environmental sign).
\end{click}
\end{frame}

\begin{frame}{Simulation}
\begin{columns}[onlytextwidth]
\begin{column}{.45\textwidth}
The image shows 500 simulated differences ($\hat{p}_{sn} - \hat{p}_e$) assuming no association between the signs and guests opting for towel re--use. The study produced $\hat{p}_{sn} - \hat{p}_e \approx 0.09$.
\end{column}
\begin{column}{.65\textwidth}
\pic{towelsimapplet.PNG}{0.93}
\end{column}
\end{columns}
\question What is the size of the $p$--value? %against the alternative $H_a: \pi_{sn} - \pi_e > 0$?
\begin{click}
   \item It will be small, 0.09 is one SD away.
   \item It will be small, 0.09 is in the tail of the null distribution. %Correct
   \item It will not be small, the null distribution is centered near 0.
   \item It will not be small, 0.09 is large compared to zero.
   \item None of the above.
\end{click}
\end{frame}

\begin{frame}{Standardized Statistic}

\begin{columns}[onlytextwidth]
\begin{column}{.45\textwidth}
\question What is true of the $z$ statistic in this context?

\begin{click}
	\item It is $\dfrac{-0.001 - 0.09}{0.045}$.
    \item It is $\dfrac{0.09 - (-0.001)}{0.045/\sqrt{433}}$.
    \item It measures the distance the statistic is from 0.5.
    \item Two of the above are true.
    \item None of the above. %Correct
\end{click}
\end{column}

\begin{column}{.65\textwidth}
\pic{towelsimapplet.PNG}{0.93}
\end{column}
\end{columns}

\end{frame}

\begin{frame}{2SD Confidence Interval for $\pi_{sn} - \pi_{e}$}

\begin{columns}[onlytextwidth]
\begin{column}{.5\textwidth}
\question What is true of the 2SD confidence interval for the parameter of interest?

\begin{click}
	\item It is not centered at 0. %Correct
    \item It has a width of 0.09.
    \item We have strong evidence that the proportion of guests that re-used towels that were exposed to the social norm sign is larger than the proportion of those exposed to the environmental sign.
    \item All of the above are true.
    \item Exactly two of A-C are true. 
\end{click}
\end{column}

\begin{column}{.65\textwidth}
\pic{towelsimapplet.PNG}{0.93}
\end{column}
\end{columns}

\end{frame}

\begin{frame}{Theory-Based Validity Conditions}
\question Are the validity conditions met in this example to trust theory-based approaches?

\begin{click}
	\item Yes, $\hat{p}_{sn}$ and $\hat{p}_e$ are far apart.
    \item Yes, $n_{sn}$ and $n_{e}$ are both larger than 20.
    \item No, $\hat{p}_{sn}$ and $\hat{p}_e$ do not equal 0.5.
    \item No, the response was not randomly assigned to the two levels of the explanatory variable.
    \item None of the above. %Correct, all of the counts in the table are at least 10.
\end{click}

\end{frame}

\begin{frame}{Theory-Based $p$-value}
\question Assuming that the validity conditions have been met, use the Theory-Based Inference applet to determine the $p$-value corresponding to the problem statement. 

\begin{click}
	\item 0.0269 %Correct
    \item 0.091
    \item 0.501
    \item 0.9731
    \item None of the above.
\end{click}

\end{frame}


%\begin{frame}{Key Terms and Ideas to Understand in Section }
%\begin{multicols}{2}
%\begin{itemize}
%	\item 
%\end{itemize}
%\end{multicols}
%\end{frame}

\end{document}