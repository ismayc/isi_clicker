\documentclass[13pt]{beamer}

\usetheme{Copenhagen}

\usecolortheme[rgb={0.60884,0.1590909,0.23106061}]{structure} % Red colors
\usefonttheme{serif}
\setbeamerfont{frametitle}{size=\normalsize}

\setbeamertemplate{navigation symbols}{}%remove navigation symbols

\usepackage[english]{babel}
\usepackage[utf8x]{inputenc}
\usepackage{multicol}
\usepackage{fmtcount}

\newcounter{count}
\addtocounter{count}{1}

\newcommand{\quotes}[2]{\centering \Large{``#1"\\
\vspace*{0.2in}
\hspace*{0.5in} - #2}}

\newcommand{\question}{ \textbf{(\decimal{count})} \stepcounter{count}}
\newcommand{\pic}[2]{\hfill\includegraphics[scale=#2]{#1}\hspace*{\fill}}

\newenvironment{stepenumerate}{\begin{enumerate}[<+->]}{\end{enumerate}}
\newenvironment{stepitemize}{\begin{itemize}[<+->]}{\end{itemize} }

\title{Sections 1.3 and 1.4}
\author{Chester Ismay, Tom Linton}
\institute{Ripon College, Central College}
\date{}


\begin{document}

\begin{frame}
  \titlepage
\end{frame}


\begin{frame}{Learning Quote of the Day}
\quotes{Trying to solve a problem before being taught the solution leads to better learning, even when errors are made in the attempt.}{Peter C. Brown, \textit{Make It Stick}}
\end{frame}

\begin{frame}{Strength of Evidence}
\question Which standardized statistic (standardized sample proportion) gives you the strongest evidence against the null hypothesis for a two-tailed test?
\begin{enumerate}[A]
	\item $z = 1$
    \item $z = 0$
    \item $z = -3$ %Correct
    \item $z = -1.8$
    \item $z = 2.9$
\end{enumerate}

\end{frame}

\begin{frame}{True Statement}
\question Identify the TRUE statement below.
\begin{enumerate}[A]
	\item As a $p$-value gets smaller, its corresponding standardized statistic gets closer to zero.
    \item Large $\hat{p}$ values always correspond with large $p$-values.
    \item A $p$-value can be negative.
  	\item A standardized statistic can be negative. %Correct
    \item We run tests of significance to determine whether $\pi$ is statistically significant.
\end{enumerate}

\end{frame}

\begin{frame}{Interpretting the $z$-score}
\question Suppose that a standardized statistic (standardized sample proportion) for a study is calculated to be $z = -2.45$. Which of the following is the most appropriate interpretation of $z$?
\begin{enumerate}[A]
	\item The observed value of the sample proportion is 2.45 SDs above the hypothesized parameter value.
    \item The observed value of the sample proportion is 2.45 times the SD of the null distribution.
    \item The observed value of the sample proportion is 2.45 times the hypothesized parameter value.
  	\item The observed value of the sample proportion is 2.45 SDs away from the hypothesized parameter value. %Correct
    \item The study results are not statistically significant.
\end{enumerate}

\end{frame}

\begin{frame}{Racquet Spinning}
Researchers wanted to investigate whether a spun tennis racquet is equally
likely to land with the label facing up or down.
\question Does this racquet spinning study call for a one-sided or a two-sided alternative?
\begin{enumerate}[A]
	\item One-sided - there is only one variable: how the label lands
  \item Two-sided - there are 2 possible outcomes: up or down
  \item One-sided - the researchers want to know whether the label is more likely to land face up
  \item Two-sided - the researchers want to know whether the spinning process is fair or biased in either direction %Correct
\end{enumerate}
\end{frame}

% \begin{frame}{Racquet Spinning (continued)}
% \question Which of the following will always be true about the standardized statistic for the racquet spinning study?
% \begin{enumerate}[A]
% 	\item The standardized statistic increases as the sample proportion that land ``up” increases. %Correct
%   \item The standardized statistic decreases as the sample proportion that land ``up” increases.
%   \item The standardized statistic increases as the sample proportion that land ``up” gets farther from 0.5.
%   \item The standardized statistic decreases as the sample proportion that land ``up” gets farther from 0.5.
% \end{enumerate}
% \end{frame}

% \begin{frame}{Racquet Spinning (continued)}
% \question Which of the following will always be true about the $p$-value for the racquet spinning study?
% \begin{enumerate}[A]
%   \item The $p$-value increases as the sample proportion that land ``up” increases.
%   \item The $p$-value decreases as the sample proportion that land ``up” increases.
%   \item The $p$-value increases as the sample proportion that land ``up” gets farther from 0.5.
%   \item The $p$-value decreases as the sample proportion that land ``up” gets farther from 0.5. %Correct
% \end{enumerate}
% \end{frame}

\begin{frame}{Setup}
Many football bets include a \textquotedblleft
point-spread\textquotedblright\ so that the team that is favored needs to win by more than that amount for a \textquotedblleft
victory.\textquotedblright\ The point-spreads are designed by professional odds makers with the intention that the probability of the favored team winning by the required amount is 0.50. In a department that has a weekly pool, where members try to predict whether or not the favored team will \textquotedblleft beat the spread," Tom correctly predicted the \textquotedblleft point-spread victor\textquotedblright\ in 8 of 12 games (so $\hat{p}=\frac{8}{12}\approx 0.67$. Is this statistically significant evidence that Tom's probability of predicting a point-spread winner is larger than 50\%?
\end{frame}

\begin{frame}
\question Shown below are 100 simulated values of $\hat{p}$ assuming $\pi=.5$ and $n=12$. Which of the following could be the $p$--value for our test of significance, $H_0:\pi=.5$ against $H_a:\pi>.5$?
\begin{multicols}{2}
\begin{enumerate}[A]
   \item 0.93
   \item 0.14
   \item 0.21 % Correct
   \item 0.50
   \item 0.67
\end{enumerate}
\pic{pointspreadsim.PNG}{.9}
\end{multicols}
\end{frame}

\begin{frame}
\question Shown below are 100 simulated values of $\hat{p}$ assuming $\pi=.5$ and $n=12$. The summary statistics give $mean = 0.496$ and $SD = 0.149$. Which of the following is the (approximate) $z$--score for our study statistic $\hat{p}=.667$?
\begin{multicols}{2}
\begin{enumerate}[A]
   \item $z=-2.66$%calculator mistake .667 - (.496 / .149)
   \item $z=0$
   \item $z=.027$%(.5 - .496) / .149
   \item $z=1.04$%subtract s divide by mu
   \item $z=1.15$%Correct 
\end{enumerate}
\pic{pointspreadsim.PNG}{.9}
\end{multicols}
\end{frame}

\begin{frame}{Comparing Evidence}
\question Similar to Exploration 1.2, you decide to test if people will pick tap water less often than expected ($H_0:\pi=1/3$ against $H_a:\pi<1/3$). You ask $n=30$ people to select the best tasting water (from three choices) and let $X=$ the number who chose tap water. Which of the following results gives the strongest evidence against the null hypothesis?
\begin{enumerate}[A]
   \item $p=.25$%
   \item $\hat{p}=\frac{13}{30}$%
   \item $X=9$ people chose tap water%
   \item $z=-2.32$%Correct
   \item $z=3$
\end{enumerate}
\end{frame}

\begin{frame}{Two Tailed $z$--scores}
\question For Doris and Buzz ($H_0:\pi=1/2$ against $H_a:\pi>1/2$, with $n=16$ trials), the $z$--score for $\hat{p}=\frac{15}{16}$ is $z=3.5$. What happens to this $z$--score if the alternative is $H_a:\pi\neq 1/2$?
\begin{enumerate}[A]
   \item It does not change.%Correct
   \item It doubles.%
   \item It gets cut in half.%
   \item You cannot tell.%
\end{enumerate}
\end{frame}

\begin{frame}{Match Results to $z$--scores}
\question For Doris and Buzz ($H_0:\pi=1/2$ against $H_a:\pi>1/2$, with $n=16$ trials), each of the $z$--scores on the right corresponds to one of the descriptions on the left. (Remember that the $z$--score for $\hat{p} = 15/16$ is $z = 3.5$.)

\begin{multicols}{2}
\begin{enumerate}
   \item Buzz got 12 correct.%Correct
   \item $\hat{p}=\frac{8}{16}$.% 
   \item $p=0.773$.% 
   \item $p=0.0013$.% 
%\end{enumerate}

\begin{enumerate}
   \item $z=-0.5$
   \item $z=0$
   \item $z=2.0$
   \item $z=3.0$
\end{enumerate}
\end{enumerate}
\end{multicols}
Which description on the left corresponds to
$z=2.0$?
%$z=0$?
%$z=2.0$?
%$z=-0.5$?
\begin{enumerate}[A]
   \item Description 1
   \item Description 2
   \item Description 3
   \item Description 4
\end{enumerate}

\end{frame}

\begin{frame}{Key Terms and Ideas to Understand in Section 1.5}
%\begin{multicols}{2}
\begin{itemize}
	\item Normal distribution
    \item Theory-based approach
    \item Validity conditions for theory-based approach
\end{itemize}
%\end{multicols}
\end{frame}


\end{document}